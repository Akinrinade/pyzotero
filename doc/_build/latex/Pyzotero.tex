% Generated by Sphinx.
\def\sphinxdocclass{report}
\documentclass[letterpaper,10pt,english]{sphinxmanual}
\usepackage[utf8]{inputenc}
\DeclareUnicodeCharacter{00A0}{\nobreakspace}
\usepackage[T1]{fontenc}
\usepackage{babel}
\usepackage{times}
\usepackage[Bjarne]{fncychap}
\usepackage{longtable}
\usepackage{sphinx}


\title{Pyzotero Documentation}
\date{July 05, 2011}
\release{0.8.2}
\author{Stephan Hügel}
\newcommand{\sphinxlogo}{}
\renewcommand{\releasename}{Release}
\makeindex

\makeatletter
\def\PYG@reset{\let\PYG@it=\relax \let\PYG@bf=\relax%
    \let\PYG@ul=\relax \let\PYG@tc=\relax%
    \let\PYG@bc=\relax \let\PYG@ff=\relax}
\def\PYG@tok#1{\csname PYG@tok@#1\endcsname}
\def\PYG@toks#1+{\ifx\relax#1\empty\else%
    \PYG@tok{#1}\expandafter\PYG@toks\fi}
\def\PYG@do#1{\PYG@bc{\PYG@tc{\PYG@ul{%
    \PYG@it{\PYG@bf{\PYG@ff{#1}}}}}}}
\def\PYG#1#2{\PYG@reset\PYG@toks#1+\relax+\PYG@do{#2}}

\def\PYG@tok@gd{\def\PYG@tc##1{\textcolor[rgb]{0.63,0.00,0.00}{##1}}}
\def\PYG@tok@gu{\let\PYG@bf=\textbf\def\PYG@tc##1{\textcolor[rgb]{0.50,0.00,0.50}{##1}}}
\def\PYG@tok@gt{\def\PYG@tc##1{\textcolor[rgb]{0.00,0.25,0.82}{##1}}}
\def\PYG@tok@gs{\let\PYG@bf=\textbf}
\def\PYG@tok@gr{\def\PYG@tc##1{\textcolor[rgb]{1.00,0.00,0.00}{##1}}}
\def\PYG@tok@cm{\let\PYG@it=\textit\def\PYG@tc##1{\textcolor[rgb]{0.25,0.50,0.56}{##1}}}
\def\PYG@tok@vg{\def\PYG@tc##1{\textcolor[rgb]{0.73,0.38,0.84}{##1}}}
\def\PYG@tok@m{\def\PYG@tc##1{\textcolor[rgb]{0.13,0.50,0.31}{##1}}}
\def\PYG@tok@mh{\def\PYG@tc##1{\textcolor[rgb]{0.13,0.50,0.31}{##1}}}
\def\PYG@tok@cs{\def\PYG@tc##1{\textcolor[rgb]{0.25,0.50,0.56}{##1}}\def\PYG@bc##1{\colorbox[rgb]{1.00,0.94,0.94}{##1}}}
\def\PYG@tok@ge{\let\PYG@it=\textit}
\def\PYG@tok@vc{\def\PYG@tc##1{\textcolor[rgb]{0.73,0.38,0.84}{##1}}}
\def\PYG@tok@il{\def\PYG@tc##1{\textcolor[rgb]{0.13,0.50,0.31}{##1}}}
\def\PYG@tok@go{\def\PYG@tc##1{\textcolor[rgb]{0.19,0.19,0.19}{##1}}}
\def\PYG@tok@cp{\def\PYG@tc##1{\textcolor[rgb]{0.00,0.44,0.13}{##1}}}
\def\PYG@tok@gi{\def\PYG@tc##1{\textcolor[rgb]{0.00,0.63,0.00}{##1}}}
\def\PYG@tok@gh{\let\PYG@bf=\textbf\def\PYG@tc##1{\textcolor[rgb]{0.00,0.00,0.50}{##1}}}
\def\PYG@tok@ni{\let\PYG@bf=\textbf\def\PYG@tc##1{\textcolor[rgb]{0.84,0.33,0.22}{##1}}}
\def\PYG@tok@nl{\let\PYG@bf=\textbf\def\PYG@tc##1{\textcolor[rgb]{0.00,0.13,0.44}{##1}}}
\def\PYG@tok@nn{\let\PYG@bf=\textbf\def\PYG@tc##1{\textcolor[rgb]{0.05,0.52,0.71}{##1}}}
\def\PYG@tok@no{\def\PYG@tc##1{\textcolor[rgb]{0.38,0.68,0.84}{##1}}}
\def\PYG@tok@na{\def\PYG@tc##1{\textcolor[rgb]{0.25,0.44,0.63}{##1}}}
\def\PYG@tok@nb{\def\PYG@tc##1{\textcolor[rgb]{0.00,0.44,0.13}{##1}}}
\def\PYG@tok@nc{\let\PYG@bf=\textbf\def\PYG@tc##1{\textcolor[rgb]{0.05,0.52,0.71}{##1}}}
\def\PYG@tok@nd{\let\PYG@bf=\textbf\def\PYG@tc##1{\textcolor[rgb]{0.33,0.33,0.33}{##1}}}
\def\PYG@tok@ne{\def\PYG@tc##1{\textcolor[rgb]{0.00,0.44,0.13}{##1}}}
\def\PYG@tok@nf{\def\PYG@tc##1{\textcolor[rgb]{0.02,0.16,0.49}{##1}}}
\def\PYG@tok@si{\let\PYG@it=\textit\def\PYG@tc##1{\textcolor[rgb]{0.44,0.63,0.82}{##1}}}
\def\PYG@tok@s2{\def\PYG@tc##1{\textcolor[rgb]{0.25,0.44,0.63}{##1}}}
\def\PYG@tok@vi{\def\PYG@tc##1{\textcolor[rgb]{0.73,0.38,0.84}{##1}}}
\def\PYG@tok@nt{\let\PYG@bf=\textbf\def\PYG@tc##1{\textcolor[rgb]{0.02,0.16,0.45}{##1}}}
\def\PYG@tok@nv{\def\PYG@tc##1{\textcolor[rgb]{0.73,0.38,0.84}{##1}}}
\def\PYG@tok@s1{\def\PYG@tc##1{\textcolor[rgb]{0.25,0.44,0.63}{##1}}}
\def\PYG@tok@gp{\let\PYG@bf=\textbf\def\PYG@tc##1{\textcolor[rgb]{0.78,0.36,0.04}{##1}}}
\def\PYG@tok@sh{\def\PYG@tc##1{\textcolor[rgb]{0.25,0.44,0.63}{##1}}}
\def\PYG@tok@ow{\let\PYG@bf=\textbf\def\PYG@tc##1{\textcolor[rgb]{0.00,0.44,0.13}{##1}}}
\def\PYG@tok@sx{\def\PYG@tc##1{\textcolor[rgb]{0.78,0.36,0.04}{##1}}}
\def\PYG@tok@bp{\def\PYG@tc##1{\textcolor[rgb]{0.00,0.44,0.13}{##1}}}
\def\PYG@tok@c1{\let\PYG@it=\textit\def\PYG@tc##1{\textcolor[rgb]{0.25,0.50,0.56}{##1}}}
\def\PYG@tok@kc{\let\PYG@bf=\textbf\def\PYG@tc##1{\textcolor[rgb]{0.00,0.44,0.13}{##1}}}
\def\PYG@tok@c{\let\PYG@it=\textit\def\PYG@tc##1{\textcolor[rgb]{0.25,0.50,0.56}{##1}}}
\def\PYG@tok@mf{\def\PYG@tc##1{\textcolor[rgb]{0.13,0.50,0.31}{##1}}}
\def\PYG@tok@err{\def\PYG@bc##1{\fcolorbox[rgb]{1.00,0.00,0.00}{1,1,1}{##1}}}
\def\PYG@tok@kd{\let\PYG@bf=\textbf\def\PYG@tc##1{\textcolor[rgb]{0.00,0.44,0.13}{##1}}}
\def\PYG@tok@ss{\def\PYG@tc##1{\textcolor[rgb]{0.32,0.47,0.09}{##1}}}
\def\PYG@tok@sr{\def\PYG@tc##1{\textcolor[rgb]{0.14,0.33,0.53}{##1}}}
\def\PYG@tok@mo{\def\PYG@tc##1{\textcolor[rgb]{0.13,0.50,0.31}{##1}}}
\def\PYG@tok@mi{\def\PYG@tc##1{\textcolor[rgb]{0.13,0.50,0.31}{##1}}}
\def\PYG@tok@kn{\let\PYG@bf=\textbf\def\PYG@tc##1{\textcolor[rgb]{0.00,0.44,0.13}{##1}}}
\def\PYG@tok@o{\def\PYG@tc##1{\textcolor[rgb]{0.40,0.40,0.40}{##1}}}
\def\PYG@tok@kr{\let\PYG@bf=\textbf\def\PYG@tc##1{\textcolor[rgb]{0.00,0.44,0.13}{##1}}}
\def\PYG@tok@s{\def\PYG@tc##1{\textcolor[rgb]{0.25,0.44,0.63}{##1}}}
\def\PYG@tok@kp{\def\PYG@tc##1{\textcolor[rgb]{0.00,0.44,0.13}{##1}}}
\def\PYG@tok@w{\def\PYG@tc##1{\textcolor[rgb]{0.73,0.73,0.73}{##1}}}
\def\PYG@tok@kt{\def\PYG@tc##1{\textcolor[rgb]{0.56,0.13,0.00}{##1}}}
\def\PYG@tok@sc{\def\PYG@tc##1{\textcolor[rgb]{0.25,0.44,0.63}{##1}}}
\def\PYG@tok@sb{\def\PYG@tc##1{\textcolor[rgb]{0.25,0.44,0.63}{##1}}}
\def\PYG@tok@k{\let\PYG@bf=\textbf\def\PYG@tc##1{\textcolor[rgb]{0.00,0.44,0.13}{##1}}}
\def\PYG@tok@se{\let\PYG@bf=\textbf\def\PYG@tc##1{\textcolor[rgb]{0.25,0.44,0.63}{##1}}}
\def\PYG@tok@sd{\let\PYG@it=\textit\def\PYG@tc##1{\textcolor[rgb]{0.25,0.44,0.63}{##1}}}

\def\PYGZbs{\char`\\}
\def\PYGZus{\char`\_}
\def\PYGZob{\char`\{}
\def\PYGZcb{\char`\}}
\def\PYGZca{\char`\^}
\def\PYGZsh{\char`\#}
\def\PYGZpc{\char`\%}
\def\PYGZdl{\char`\$}
\def\PYGZti{\char`\~}
% for compatibility with earlier versions
\def\PYGZat{@}
\def\PYGZlb{[}
\def\PYGZrb{]}
\makeatother

\begin{document}

\maketitle
\tableofcontents
\phantomsection\label{index::doc}


A Python wrapper for the \href{http://www.zotero.org/support/dev/server\_api}{Zotero API}. You'll require a user ID and access key, which can be set up \href{http://www.zotero.org/settings/keys/new}{here}.
\phantomsection\label{index:module-pyzotero.zotero}
\index{pyzotero.zotero (module)}

\chapter{Indices and tables}
\label{index:indices-and-tables}\label{index:description}\begin{itemize}
\item {} 
\emph{genindex}

\item {} 
\emph{modindex}

\item {} 
\emph{search}

\end{itemize}


\chapter{Installation}
\label{index:installation}
Using pip: \code{pip install pyzotero}

From a local clone, if you wish to install Pyzotero from a specific branch:

\begin{Verbatim}[commandchars=@\[\]]
git clone git://github.com/urschrei/pyzotero.git
cd pyzotero
git checkout dev
pip install .
\end{Verbatim}

Alternatively, download the latest version from \href{https://github.com/urschrei/pyzotero/downloads}{https://github.com/urschrei/pyzotero/downloads}, and point pip at the zip file.
Example: \code{pip install \textasciitilde{}/Downloads/urschrei-pyzotero-v0.3-0-g04ff544.zip}

I assume that running setup.py will also work using \code{easy\_install}, but I haven't tested it.

The \href{http://feedparser.org}{feedparser} module (\textgreater{}= 0.5.1) is required. It should automatically be installed when installing Pyzotero using pip.


\section{Testing}
\label{index:testing}
Run \code{tests.py} in the pyzotero directory, or, using \href{http://somethingaboutorange.com/mrl/projects/nose/1.0.0/index.html}{Nose}, \code{nosetests} from this directory. If you wish to see coverage statistics, run \code{nosetests -{-}with-coverage -{-}cover-package=pyzotero}.


\chapter{Usage}
\label{index:usage}

\section{Hello World}
\label{index:hello-world}\label{index:id1}
\begin{Verbatim}[commandchars=\\\{\}]
\PYG{k+kn}{from} \PYG{n+nn}{pyzotero} \PYG{k+kn}{import} \PYG{n}{zotero}
\PYG{n}{zot} \PYG{o}{=} \PYG{n}{zotero}\PYG{o}{.}\PYG{n}{Zotero}\PYG{p}{(}\PYG{n}{user\PYGZus{}id}\PYG{p}{,} \PYG{n}{user\PYGZus{}key}\PYG{p}{)}
\PYG{n}{items} \PYG{o}{=} \PYG{n}{zot}\PYG{o}{.}\PYG{n}{items}\PYG{p}{(}\PYG{p}{)}
\PYG{k}{for} \PYG{n}{item} \PYG{o+ow}{in} \PYG{n}{items}\PYG{p}{:}
    \PYG{k}{print} \PYG{l+s}{'}\PYG{l+s}{Author: }\PYG{l+s+si}{\PYGZpc{}s}\PYG{l+s}{ \textbar{} Title: }\PYG{l+s+si}{\PYGZpc{}s}\PYG{l+s}{'} \PYG{o}{\PYGZpc{}} \PYG{p}{(}\PYG{n}{item}\PYG{p}{[}\PYG{l+s}{'}\PYG{l+s}{creators}\PYG{l+s}{'}\PYG{p}{]}\PYG{p}{[}\PYG{l+m+mi}{0}\PYG{p}{]}\PYG{p}{[}\PYG{l+s}{'}\PYG{l+s}{lastName}\PYG{l+s}{'}\PYG{p}{]}\PYG{p}{,} \PYG{n}{item}\PYG{p}{[}\PYG{l+s}{'}\PYG{l+s}{title}\PYG{l+s}{'}\PYG{p}{]}\PYG{p}{)}
\end{Verbatim}


\section{General Usage}
\label{index:general-usage}
First, create a new Zotero instance:
\begin{quote}

\index{Zotero (class in pyzotero.zotero)}

\begin{fulllineitems}
\phantomsection\label{index:pyzotero.zotero.Zotero}\pysiglinewithargsret{\strong{class }\code{pyzotero.zotero.}\bfcode{Zotero}}{\emph{userID}, \emph{userKey}}{}~\begin{quote}\begin{description}
\item[{Parameters}] \leavevmode\begin{itemize}
\item {} 
\textbf{userID} (\emph{str}) -- a valid Zotero API user ID

\item {} 
\textbf{userKey} (\emph{str}) -- a valid Zotero API user key

\end{itemize}

\end{description}\end{quote}

Example:

\begin{Verbatim}[commandchars=\\\{\}]
\PYG{n}{zot} \PYG{o}{=} \PYG{n}{zotero}\PYG{o}{.}\PYG{n}{Zotero}\PYG{p}{(}\PYG{l+m+mi}{123}\PYG{p}{,} \PYG{n}{ABC1234XYZ}\PYG{p}{)}
\end{Verbatim}

\end{fulllineitems}

\end{quote}

Additional parameters may be set on Read API methods using the following method. They are \textbf{optional}:
\begin{quote}

\index{add\_parameters() (pyzotero.zotero.Zotero method)}

\begin{fulllineitems}
\phantomsection\label{index:pyzotero.zotero.Zotero.add_parameters}\pysiglinewithargsret{\code{Zotero.}\bfcode{add\_parameters}}{\optional{\emph{limit=None}, \emph{start=None}, \emph{order=None}, \emph{sort=None}\optional{, \emph{content=None}\optional{, \emph{style=None}}}}}{}~\begin{quote}\begin{description}
\item[{Parameters}] \leavevmode\begin{itemize}
\item {} 
\textbf{limit} (\emph{int}) -- 1 – 99 or None

\item {} 
\textbf{start} (\emph{int}) -- 1 – total number of items in your library or None

\item {} 
\textbf{order} (\emph{str}) -- A valid Zotero field or None

\item {} 
\textbf{sort} (\emph{str}) -- asc or desc or None

\item {} 
\textbf{content} (\emph{str}) -- `html' or `bib', default: `html'. If `bib' is passed, you may also pass:

\item {} 
\textbf{style} (\emph{str}) -- Any valid CSL style in the Zotero style repository

\end{itemize}

\item[{Return type}] \leavevmode
list of HTML strings or None

\end{description}\end{quote}

Example:

\begin{Verbatim}[commandchars=\\\{\}]
\PYG{n}{zot}\PYG{o}{.}\PYG{n}{add\PYGZus{}parameters}\PYG{p}{(}\PYG{n}{limit}\PYG{o}{=}\PYG{l+m+mi}{7}\PYG{p}{,}\PYG{n}{start}\PYG{o}{=}\PYG{l+m+mi}{3}\PYG{p}{)}
\end{Verbatim}

\end{fulllineitems}

\end{quote}

\begin{notice}{note}{Note:}
Any parameters you set will be valid \textbf{for the next call only}
\end{notice}

The special parameters \code{content} and \code{style}:
\begin{quote}

Example:

\begin{Verbatim}[commandchars=\\\{\}]
\PYG{n}{zot}\PYG{o}{.}\PYG{n}{add\PYGZus{}parameters}\PYG{p}{(}\PYG{n}{content}\PYG{o}{=}\PYG{l+s}{'}\PYG{l+s}{bib}\PYG{l+s}{'}\PYG{p}{,} \PYG{n}{format}\PYG{o}{=}\PYG{l+s}{'}\PYG{l+s}{mla}\PYG{l+s}{'}\PYG{p}{)}
\end{Verbatim}

If these are set, the return value is a \textbf{list} of UTF-8 formatted HTML \code{div} elements, each containing an item:

\code{{[}'\textless{}div class="csl-entry"\textgreater{}(content)\textless{}/div\textgreater{}', … {]}}
\end{quote}


\chapter{Read API Methods}
\label{index:read-api-methods}

\section{Retrieving Items}
\label{index:retrieving-items}\begin{quote}

\index{items() (pyzotero.zotero.Zotero method)}

\begin{fulllineitems}
\phantomsection\label{index:pyzotero.zotero.Zotero.items}\pysiglinewithargsret{\code{Zotero.}\bfcode{items}}{}{}
Returns Zotero library items
\begin{quote}\begin{description}
\item[{Return type}] \leavevmode
list of dicts

\end{description}\end{quote}

\end{fulllineitems}


\index{top() (pyzotero.zotero.Zotero method)}

\begin{fulllineitems}
\phantomsection\label{index:pyzotero.zotero.Zotero.top}\pysiglinewithargsret{\code{Zotero.}\bfcode{top}}{}{}
Returns top-level Zotero library items
\begin{quote}\begin{description}
\item[{Return type}] \leavevmode
list of dicts

\end{description}\end{quote}

\end{fulllineitems}


\index{item() (pyzotero.zotero.Zotero method)}

\begin{fulllineitems}
\phantomsection\label{index:pyzotero.zotero.Zotero.item}\pysiglinewithargsret{\code{Zotero.}\bfcode{item}}{\emph{itemID}}{}
Returns a specific item
\begin{quote}\begin{description}
\item[{Parameters}] \leavevmode
\textbf{itemID} (\emph{str}) -- a zotero item ID

\item[{Return type}] \leavevmode
list of dicts

\end{description}\end{quote}

\end{fulllineitems}


\index{children() (pyzotero.zotero.Zotero method)}

\begin{fulllineitems}
\phantomsection\label{index:pyzotero.zotero.Zotero.children}\pysiglinewithargsret{\code{Zotero.}\bfcode{children}}{\emph{itemID}}{}
Returns the child items of a specific item
\begin{quote}\begin{description}
\item[{Parameters}] \leavevmode
\textbf{itemID} (\emph{str}) -- a zotero item ID

\item[{Return type}] \leavevmode
list of dicts

\end{description}\end{quote}

\end{fulllineitems}


\index{tag\_items() (pyzotero.zotero.Zotero method)}

\begin{fulllineitems}
\phantomsection\label{index:pyzotero.zotero.Zotero.tag_items}\pysiglinewithargsret{\code{Zotero.}\bfcode{tag\_items}}{\emph{itemID}}{}
Returns items for a specific tag
\begin{quote}\begin{description}
\item[{Parameters}] \leavevmode
\textbf{itemID} (\emph{str}) -- a zotero item ID

\item[{Return type}] \leavevmode
list of dicts

\end{description}\end{quote}

\end{fulllineitems}


\index{group\_items() (pyzotero.zotero.Zotero method)}

\begin{fulllineitems}
\phantomsection\label{index:pyzotero.zotero.Zotero.group_items}\pysiglinewithargsret{\code{Zotero.}\bfcode{group\_items}}{\emph{groupID}}{}
Returns items from a specific group
\begin{quote}\begin{description}
\item[{Parameters}] \leavevmode
\textbf{groupID} (\emph{str}) -- a Zotero group ID

\item[{Return type}] \leavevmode
list of dicts

\end{description}\end{quote}

\end{fulllineitems}


\index{group\_top() (pyzotero.zotero.Zotero method)}

\begin{fulllineitems}
\phantomsection\label{index:pyzotero.zotero.Zotero.group_top}\pysiglinewithargsret{\code{Zotero.}\bfcode{group\_top}}{\emph{groupID}}{}
Returns top-level items from a specific group
\begin{quote}\begin{description}
\item[{Parameters}] \leavevmode
\textbf{groupID} (\emph{str}) -- a Zotero group ID

\item[{Return type}] \leavevmode
list of dicts

\end{description}\end{quote}

\end{fulllineitems}


\index{group\_item() (pyzotero.zotero.Zotero method)}

\begin{fulllineitems}
\phantomsection\label{index:pyzotero.zotero.Zotero.group_item}\pysiglinewithargsret{\code{Zotero.}\bfcode{group\_item}}{\emph{groupID}, \emph{itemID}}{}
Returns a specific item from a specific group
\begin{quote}\begin{description}
\item[{Parameters}] \leavevmode\begin{itemize}
\item {} 
\textbf{groupID} (\emph{str}) -- a Zotero group ID

\item {} 
\textbf{itemID} (\emph{str}) -- a zotero item ID

\end{itemize}

\item[{Return type}] \leavevmode
list of dicts

\end{description}\end{quote}

\end{fulllineitems}


\index{group\_item\_children() (pyzotero.zotero.Zotero method)}

\begin{fulllineitems}
\phantomsection\label{index:pyzotero.zotero.Zotero.group_item_children}\pysiglinewithargsret{\code{Zotero.}\bfcode{group\_item\_children}}{\emph{groupID}, \emph{itemID}}{}
Returns the child items of a specific item from a specific group
\begin{quote}\begin{description}
\item[{Parameters}] \leavevmode\begin{itemize}
\item {} 
\textbf{groupID} (\emph{str}) -- a Zotero group ID

\item {} 
\textbf{itemID} (\emph{str}) -- a Zotero item ID

\end{itemize}

\item[{Return type}] \leavevmode
list of dicts

\end{description}\end{quote}

\end{fulllineitems}


\index{group\_items\_tag() (pyzotero.zotero.Zotero method)}

\begin{fulllineitems}
\phantomsection\label{index:pyzotero.zotero.Zotero.group_items_tag}\pysiglinewithargsret{\code{Zotero.}\bfcode{group\_items\_tag}}{\emph{groupID}, \emph{tag}}{}
Returns a specific group's items for a specific tag
\begin{quote}\begin{description}
\item[{Parameters}] \leavevmode\begin{itemize}
\item {} 
\textbf{groupID} (\emph{str}) -- a Zotero group ID

\item {} 
\textbf{tag} (\emph{str}) -- a tag whose items you wish to return

\end{itemize}

\item[{Return type}] \leavevmode
list of dicts

\end{description}\end{quote}

\end{fulllineitems}


\index{group\_collection\_items() (pyzotero.zotero.Zotero method)}

\begin{fulllineitems}
\phantomsection\label{index:pyzotero.zotero.Zotero.group_collection_items}\pysiglinewithargsret{\code{Zotero.}\bfcode{group\_collection\_items}}{\emph{groupID}, \emph{collection ID}}{}
Returns a specific collection's items from a specific group
\begin{quote}\begin{description}
\item[{Parameters}] \leavevmode\begin{itemize}
\item {} 
\textbf{groupID} (\emph{str}) -- a Zotero group ID

\item {} 
\textbf{collectionID} (\emph{str}) -- a Zotero collection ID

\end{itemize}

\item[{Return type}] \leavevmode
list of dicts

\end{description}\end{quote}

\end{fulllineitems}


\index{group\_collection\_item() (pyzotero.zotero.Zotero method)}

\begin{fulllineitems}
\phantomsection\label{index:pyzotero.zotero.Zotero.group_collection_item}\pysiglinewithargsret{\code{Zotero.}\bfcode{group\_collection\_item}}{\emph{groupID}, \emph{collectionID}, \emph{itemID}}{}
Returns a specific collection's item from a specific group
\begin{quote}\begin{description}
\item[{Parameters}] \leavevmode\begin{itemize}
\item {} 
\textbf{groupID} (\emph{str}) -- a Zotero group ID

\item {} 
\textbf{collectionID} (\emph{str}) -- a Zotero collection ID

\item {} 
\textbf{itemID} (\emph{str}) -- a zotero item ID

\end{itemize}

\item[{Return type}] \leavevmode
list of dicts

\end{description}\end{quote}

\end{fulllineitems}


\index{group\_collection\_top() (pyzotero.zotero.Zotero method)}

\begin{fulllineitems}
\phantomsection\label{index:pyzotero.zotero.Zotero.group_collection_top}\pysiglinewithargsret{\code{Zotero.}\bfcode{group\_collection\_top}}{\emph{groupID}, \emph{collectionID}}{}
Returns a specific collection's top-level items from a specific group
\begin{quote}\begin{description}
\item[{Parameters}] \leavevmode\begin{itemize}
\item {} 
\textbf{groupID} (\emph{str}) -- a Zotero group ID

\item {} 
\textbf{groupID} -- a Zotero collection ID

\end{itemize}

\item[{Return type}] \leavevmode
list of dicts

\end{description}\end{quote}

\end{fulllineitems}


\index{collection\_items() (pyzotero.zotero.Zotero method)}

\begin{fulllineitems}
\phantomsection\label{index:pyzotero.zotero.Zotero.collection_items}\pysiglinewithargsret{\code{Zotero.}\bfcode{collection\_items}}{\emph{collectionID}}{}
Returns items from the specified collection
\begin{quote}\begin{description}
\item[{Parameters}] \leavevmode
\textbf{collectionID} (\emph{str}) -- a Zotero collection ID

\item[{Return type}] \leavevmode
list of dicts

\end{description}\end{quote}

\end{fulllineitems}


\index{get\_subset() (pyzotero.zotero.Zotero method)}

\begin{fulllineitems}
\phantomsection\label{index:pyzotero.zotero.Zotero.get_subset}\pysiglinewithargsret{\code{Zotero.}\bfcode{get\_subset}}{\emph{itemIDs}}{}
Retrieve an arbitrary set of non-adjacent items. Limited to 50 items per call.
\begin{quote}\begin{description}
\item[{Parameters}] \leavevmode
\textbf{itemIDs} (\emph{list}) -- a list of Zotero Item IDs

\item[{Return type}] \leavevmode
list of dicts

\end{description}\end{quote}

\end{fulllineitems}

\end{quote}
\phantomsection\label{index:returned}\begin{quote}

Example of returned data:

\begin{Verbatim}[commandchars=@\[\]]
@PYGZlb[]{'DOI': '',
 'ISSN': '1747-1532',
 'abstractNote': '',
 'accessDate': '',
 'archive': '',
 'archiveLocation': '',
 'callNumber': '',
 'creators': @PYGZlb[]{'creatorType': 'author',
               'firstName': 'T. J.',
               'lastName': 'McIntyre'}@PYGZrb[],
 'date': '2007',
 'extra': '',
 'issue': '',
 'itemType': 'journalArticle',
 'journalAbbreviation': '',
 'language': '',
 'libraryCatalog': 'Google Scholar',
 'pages': '',
 'publicationTitle': 'Journal of Intellectual Property Law @& Practice',
 'rights': '',
 'series': '',
 'seriesText': '',
 'seriesTitle': '',
 'shortTitle': 'Copyright in custom code',
 'tags': @PYGZlb[]@PYGZrb[],
 'title': 'Copyright in custom code: Who owns commissioned software?',
 'url': '',
 'volume': ''} … @PYGZrb[]
\end{Verbatim}

See {\hyperref[index:hello-world]{\emph{`Hello World'}}} example, above
\end{quote}


\section{Retrieving Collections}
\label{index:retrieving-collections}\begin{quote}

\index{collections() (pyzotero.zotero.Zotero method)}

\begin{fulllineitems}
\phantomsection\label{index:pyzotero.zotero.Zotero.collections}\pysiglinewithargsret{\code{Zotero.}\bfcode{collections}}{}{}
Returns a user's collections
\begin{quote}\begin{description}
\item[{Return type}] \leavevmode
list of dicts

\end{description}\end{quote}

\end{fulllineitems}


\index{collections\_sub() (pyzotero.zotero.Zotero method)}

\begin{fulllineitems}
\phantomsection\label{index:pyzotero.zotero.Zotero.collections_sub}\pysiglinewithargsret{\code{Zotero.}\bfcode{collections\_sub}}{\emph{collectionID}}{}
Returns a sub-collection from a specific collection
\begin{quote}\begin{description}
\item[{Parameters}] \leavevmode
\textbf{collectionID} (\emph{str}) -- a Zotero library collection ID

\item[{Return type}] \leavevmode
list of dicts

\end{description}\end{quote}

\end{fulllineitems}


\index{group\_collections() (pyzotero.zotero.Zotero method)}

\begin{fulllineitems}
\phantomsection\label{index:pyzotero.zotero.Zotero.group_collections}\pysiglinewithargsret{\code{Zotero.}\bfcode{group\_collections}}{\emph{groupID}}{}
Returns collections for a specific group
\begin{quote}\begin{description}
\item[{Parameters}] \leavevmode
\textbf{groupID} (\emph{str}) -- a Zotero group ID

\item[{Return type}] \leavevmode
list of dicts

\end{description}\end{quote}

\end{fulllineitems}


\index{group\_collection() (pyzotero.zotero.Zotero method)}

\begin{fulllineitems}
\phantomsection\label{index:pyzotero.zotero.Zotero.group_collection}\pysiglinewithargsret{\code{Zotero.}\bfcode{group\_collection}}{\emph{groupID}, \emph{collectionID}}{}
Returns a specific collection from a specific group
\begin{quote}\begin{description}
\item[{Parameters}] \leavevmode\begin{itemize}
\item {} 
\textbf{groupID} (\emph{str}) -- a Zotero group ID

\item {} 
\textbf{collectionID} (\emph{str}) -- a Zotero collection ID

\end{itemize}

\item[{Return type}] \leavevmode
list of dicts

\end{description}\end{quote}

\end{fulllineitems}


Example of returned data:

\begin{Verbatim}[commandchars=@\[\]]
@PYGZlb[]{'key': 'PRMD6BGB', 'name': "A Midsummer Night's Dream"} … @PYGZrb[]
\end{Verbatim}
\end{quote}


\section{Retrieving groups}
\label{index:retrieving-groups}\begin{quote}

\index{groups() (pyzotero.zotero.Zotero method)}

\begin{fulllineitems}
\phantomsection\label{index:pyzotero.zotero.Zotero.groups}\pysiglinewithargsret{\code{Zotero.}\bfcode{groups}}{}{}
Retrieve the Zotero group data to which the current user key has access
\begin{quote}\begin{description}
\item[{Return type}] \leavevmode
list of dicts

\end{description}\end{quote}

Example of returned data:

\begin{Verbatim}[commandchars=@\[\]]
@PYGZlb[]{u'description': u'@%3Cp@%3EBGerman+Cinema+and+related+literature.@%3C@%2Fp@%3E',
    u'fileEditing': u'none',
    u'group@_id': u'153',
    u'hasImage': 1,
    u'libraryEditing': u'admins',
    u'libraryEnabled': 1,
    u'libraryReading': u'all',
    u'members': {u'0': 436,
       u'1': 6972,
       u'15': 499956,
       u'16': 521307,
       u'17': 619180},
    u'name': u'German Cinema',
    u'owner': 10421,
    u'type': u'PublicOpen',
    u'url': u''} … @PYGZrb[]
\end{Verbatim}

\end{fulllineitems}

\end{quote}


\section{Retrieving Tags}
\label{index:retrieving-tags}\begin{quote}

\index{tags() (pyzotero.zotero.Zotero method)}

\begin{fulllineitems}
\phantomsection\label{index:pyzotero.zotero.Zotero.tags}\pysiglinewithargsret{\code{Zotero.}\bfcode{tags}}{}{}
Returns a user's tags
\begin{quote}\begin{description}
\item[{Return type}] \leavevmode
list of strings

\end{description}\end{quote}

\end{fulllineitems}


\index{item\_tags() (pyzotero.zotero.Zotero method)}

\begin{fulllineitems}
\phantomsection\label{index:pyzotero.zotero.Zotero.item_tags}\pysiglinewithargsret{\code{Zotero.}\bfcode{item\_tags}}{\emph{itemID}}{}
Returns tags from a specific item
\begin{quote}\begin{description}
\item[{Parameters}] \leavevmode
\textbf{itemID} (\emph{str}) -- a valid Zotero library Item ID

\item[{Return type}] \leavevmode
list of strings

\end{description}\end{quote}

\end{fulllineitems}


\index{group\_tags() (pyzotero.zotero.Zotero method)}

\begin{fulllineitems}
\phantomsection\label{index:pyzotero.zotero.Zotero.group_tags}\pysiglinewithargsret{\code{Zotero.}\bfcode{group\_tags}}{\emph{groupID}}{}
Returns tags from a specific group
\begin{quote}\begin{description}
\item[{Parameters}] \leavevmode
\textbf{groupID} (\emph{str}) -- a valid Zotero library group ID

\item[{Return type}] \leavevmode
list of strings

\end{description}\end{quote}

\end{fulllineitems}


\index{group\_item\_tags() (pyzotero.zotero.Zotero method)}

\begin{fulllineitems}
\phantomsection\label{index:pyzotero.zotero.Zotero.group_item_tags}\pysiglinewithargsret{\code{Zotero.}\bfcode{group\_item\_tags}}{\emph{groupID}, \emph{itemID}}{}
Returns tags from a specific item from a specific group
\begin{quote}\begin{description}
\item[{Parameters}] \leavevmode\begin{itemize}
\item {} 
\textbf{groupID} (\emph{str}) -- a valid Zotero library group ID

\item {} 
\textbf{itemID} (\emph{str}) -- a valid Zotero library Item ID

\end{itemize}

\item[{Return type}] \leavevmode
list of strings

\end{description}\end{quote}

\end{fulllineitems}

\end{quote}

Example of returned data:

\begin{Verbatim}[commandchars=@\[\]]
@PYGZlb[]'Authority in literature', 'Errata', … @PYGZrb[]
\end{Verbatim}


\chapter{Write API Methods}
\label{index:write-api-methods}

\section{Item Methods}
\label{index:item-methods}\begin{quote}

\index{item\_types() (pyzotero.zotero.Zotero method)}

\begin{fulllineitems}
\phantomsection\label{index:pyzotero.zotero.Zotero.item_types}\pysiglinewithargsret{\code{Zotero.}\bfcode{item\_types}}{}{}
Returns a dict containing all available item types
\begin{quote}\begin{description}
\item[{Return type}] \leavevmode
dict

\end{description}\end{quote}

\end{fulllineitems}


\index{item\_fields() (pyzotero.zotero.Zotero method)}

\begin{fulllineitems}
\phantomsection\label{index:pyzotero.zotero.Zotero.item_fields}\pysiglinewithargsret{\code{Zotero.}\bfcode{item\_fields}}{}{}
Returns a dict of all available item fields
\begin{quote}\begin{description}
\item[{Return type}] \leavevmode
dict

\end{description}\end{quote}

\end{fulllineitems}


\index{item\_creator\_types() (pyzotero.zotero.Zotero method)}

\begin{fulllineitems}
\phantomsection\label{index:pyzotero.zotero.Zotero.item_creator_types}\pysiglinewithargsret{\code{Zotero.}\bfcode{item\_creator\_types}}{\emph{itemtype}}{}
Returns a dict of all valid creator types for the specified item type
\begin{quote}\begin{description}
\item[{Parameters}] \leavevmode
\textbf{itemtype} (\emph{str}) -- a valid Zotero item type. A list of available item types can be obtained by the use of {\hyperref[index:pyzotero.zotero.Zotero.item_types]{\code{item\_types()}}}

\item[{Return type}] \leavevmode
dict

\end{description}\end{quote}

\end{fulllineitems}


\index{item\_template() (pyzotero.zotero.Zotero method)}

\begin{fulllineitems}
\phantomsection\label{index:pyzotero.zotero.Zotero.item_template}\pysiglinewithargsret{\code{Zotero.}\bfcode{item\_template}}{\emph{itemtype}}{}
Returns an item creation template for the specified item type
\begin{quote}\begin{description}
\item[{Parameters}] \leavevmode
\textbf{itemtype} (\emph{str}) -- a valid Zotero item type. A list of available item types can be obtained by the use of {\hyperref[index:pyzotero.zotero.Zotero.item_types]{\code{item\_types()}}}

\item[{Return type}] \leavevmode
dict

\end{description}\end{quote}

\end{fulllineitems}


\index{create\_item() (pyzotero.zotero.Zotero method)}

\begin{fulllineitems}
\phantomsection\label{index:pyzotero.zotero.Zotero.create_item}\pysiglinewithargsret{\code{Zotero.}\bfcode{create\_item}}{\emph{items}}{}
Create Zotero library items.
\begin{quote}\begin{description}
\item[{Parameters}] \leavevmode
\textbf{items} (\emph{list}) -- one or more dicts containing item data.

\item[{Return type}] \leavevmode
list of dicts

\end{description}\end{quote}

Returns a copy of the created item(s), if successful. The use of {\hyperref[index:pyzotero.zotero.Zotero.item_template]{\code{item\_template()}}} is recommended in order to first obtain a dict with a structure which the API will accept.
Example:

\begin{Verbatim}[commandchars=\\\{\}]
\PYG{n}{template} \PYG{o}{=} \PYG{n}{zot}\PYG{o}{.}\PYG{n}{item\PYGZus{}template}\PYG{p}{(}\PYG{l+s}{'}\PYG{l+s}{book}\PYG{l+s}{'}\PYG{p}{)}
\PYG{n}{template}\PYG{p}{[}\PYG{l+s}{'}\PYG{l+s}{creators}\PYG{l+s}{'}\PYG{p}{]}\PYG{p}{[}\PYG{l+m+mi}{0}\PYG{p}{]}\PYG{p}{[}\PYG{l+s}{'}\PYG{l+s}{firstName}\PYG{l+s}{'}\PYG{p}{]} \PYG{o}{=} \PYG{l+s}{'}\PYG{l+s}{Monty}\PYG{l+s}{'}
\PYG{n}{template}\PYG{p}{[}\PYG{l+s}{'}\PYG{l+s}{creators}\PYG{l+s}{'}\PYG{p}{]}\PYG{p}{[}\PYG{l+m+mi}{0}\PYG{p}{]}\PYG{p}{[}\PYG{l+s}{'}\PYG{l+s}{lastName}\PYG{l+s}{'}\PYG{p}{]} \PYG{o}{=} \PYG{l+s}{'}\PYG{l+s}{Cantsin}\PYG{l+s}{'}
\PYG{n}{template}\PYG{p}{[}\PYG{l+s}{'}\PYG{l+s}{title}\PYG{l+s}{'}\PYG{p}{]} \PYG{o}{=} \PYG{l+s}{'}\PYG{l+s}{Maris Kundzins: A Life}\PYG{l+s}{'}
\PYG{n}{resp} \PYG{o}{=} \PYG{n}{zot}\PYG{o}{.}\PYG{n}{create\PYGZus{}item}\PYG{p}{(}\PYG{p}{[}\PYG{n}{template}\PYG{p}{]}\PYG{p}{)}
\end{Verbatim}

If successful, \code{resp} will have the same structure as items retrieved with an {\hyperref[index:pyzotero.zotero.Zotero.items]{\code{items()}}} call, e.g. a list of one or more dicts (see {\hyperref[index:returned]{\emph{Item Data}}}, above).

\end{fulllineitems}


\index{update\_item() (pyzotero.zotero.Zotero method)}

\begin{fulllineitems}
\phantomsection\label{index:pyzotero.zotero.Zotero.update_item}\pysiglinewithargsret{\code{Zotero.}\bfcode{update\_item}}{\emph{item}}{}
Update an item in your library.
\begin{quote}\begin{description}
\item[{Parameters}] \leavevmode
\textbf{item} (\emph{dict}) -- a dict containing item data.

\item[{Return type}] \leavevmode
Boolean

\end{description}\end{quote}

Example:

\begin{Verbatim}[commandchars=\\\{\}]
\PYG{n}{i} \PYG{o}{=} \PYG{n}{zot}\PYG{o}{.}\PYG{n}{items}\PYG{p}{(}\PYG{p}{)}
\PYG{c}{\PYGZsh{} see above for example of returned item structure}
\PYG{c}{\PYGZsh{} modify the latest item which was added to your library}
\PYG{n}{i}\PYG{p}{[}\PYG{l+m+mi}{0}\PYG{p}{]}\PYG{p}{[}\PYG{l+s}{'}\PYG{l+s}{title}\PYG{l+s}{'}\PYG{p}{]} \PYG{o}{=} \PYG{l+s}{'}\PYG{l+s}{The Sheltering Sky}\PYG{l+s}{'}
\PYG{n}{i}\PYG{p}{[}\PYG{l+m+mi}{0}\PYG{p}{]}\PYG{p}{[}\PYG{l+s}{'}\PYG{l+s}{creators}\PYG{l+s}{'}\PYG{p}{]}\PYG{p}{[}\PYG{l+m+mi}{0}\PYG{p}{]}\PYG{p}{[}\PYG{l+s}{'}\PYG{l+s}{firstName}\PYG{l+s}{'}\PYG{p}{]} \PYG{o}{=} \PYG{l+s}{'}\PYG{l+s}{Paul}\PYG{l+s}{'}
\PYG{n}{i}\PYG{p}{[}\PYG{l+m+mi}{0}\PYG{p}{]}\PYG{p}{[}\PYG{l+s}{'}\PYG{l+s}{creators}\PYG{l+s}{'}\PYG{p}{]}\PYG{p}{[}\PYG{l+m+mi}{0}\PYG{p}{]}\PYG{p}{[}\PYG{l+s}{'}\PYG{l+s}{lastName}\PYG{l+s}{'}\PYG{p}{]} \PYG{o}{=} \PYG{l+s}{'}\PYG{l+s}{Bowles}\PYG{l+s}{'}
\PYG{n}{zot}\PYG{o}{.}\PYG{n}{update\PYGZus{}item}\PYG{p}{(}\PYG{n}{i}\PYG{p}{[}\PYG{l+m+mi}{0}\PYG{p}{]}\PYG{p}{)}
\end{Verbatim}

\end{fulllineitems}


\index{delete\_item() (pyzotero.zotero.Zotero method)}

\begin{fulllineitems}
\phantomsection\label{index:pyzotero.zotero.Zotero.delete_item}\pysiglinewithargsret{\code{Zotero.}\bfcode{delete\_item}}{\emph{item}}{}
Delete an item from your library.
\begin{quote}\begin{description}
\item[{Parameters}] \leavevmode
\textbf{item} (\emph{dict}) -- a dict containing item data. As in the previous example, you must first retrieve the item(s) you wish to delete, and pass it/them to the method one by one. Deletion of multiple items is most easily accomplished using e.g. a \code{for} loop.

\item[{Return type}] \leavevmode
Boolean

\end{description}\end{quote}

Example:

\begin{Verbatim}[commandchars=\\\{\}]
\PYG{n}{i} \PYG{o}{=} \PYG{n}{zot}\PYG{o}{.}\PYG{n}{items}\PYG{p}{(}\PYG{p}{)}
\PYG{c}{\PYGZsh{} only delete the last five items we added}
\PYG{n}{to\PYGZus{}delete} \PYG{o}{=} \PYG{n}{i}\PYG{p}{[}\PYG{p}{:}\PYG{l+m+mi}{6}\PYG{p}{]}
\PYG{k}{for} \PYG{n}{d} \PYG{o+ow}{in} \PYG{n}{to\PYGZus{}delete}\PYG{p}{:}
    \PYG{n}{zot}\PYG{o}{.}\PYG{n}{delete\PYGZus{}item}\PYG{p}{(}\PYG{n}{d}\PYG{p}{)}
\end{Verbatim}

\end{fulllineitems}

\end{quote}


\section{Collection Methods}
\label{index:collection-methods}\begin{quote}

\index{create\_collection() (pyzotero.zotero.Zotero method)}

\begin{fulllineitems}
\phantomsection\label{index:pyzotero.zotero.Zotero.create_collection}\pysiglinewithargsret{\code{Zotero.}\bfcode{create\_collection}}{\emph{name}}{}
Create a new collection in the Zotero library.
\begin{quote}\begin{description}
\item[{Parameters}] \leavevmode
\textbf{name} (\emph{dict}) -- dict containing the key \code{name} and the value of the new collection name you wish to create. May optionally contain a \code{parent} key, the value of which is the ID of an existing collection. If this is set, the collection will be created as a child of that collection.

\item[{Return type}] \leavevmode
Boolean

\end{description}\end{quote}

\end{fulllineitems}


\index{addto\_collection() (pyzotero.zotero.Zotero method)}

\begin{fulllineitems}
\phantomsection\label{index:pyzotero.zotero.Zotero.addto_collection}\pysiglinewithargsret{\code{Zotero.}\bfcode{addto\_collection}}{\emph{collection}, \emph{items}}{}
Add the specified item(s) to the specified collection.
\begin{quote}\begin{description}
\item[{Parameters}] \leavevmode\begin{itemize}
\item {} 
\textbf{collection} (\emph{str}) -- a collection key

\item {} 
\textbf{items} (\emph{list}) -- list of one or more item dicts.

\end{itemize}

\item[{Return type}] \leavevmode
Boolean

\end{description}\end{quote}

Collection keys can be obtained by a call to {\hyperref[index:pyzotero.zotero.Zotero.collections]{\code{collections()}}} (see details above).

\end{fulllineitems}


\index{deletefrom\_collection() (pyzotero.zotero.Zotero method)}

\begin{fulllineitems}
\phantomsection\label{index:pyzotero.zotero.Zotero.deletefrom_collection}\pysiglinewithargsret{\code{Zotero.}\bfcode{deletefrom\_collection}}{\emph{collection}, \emph{item}}{}
Remove the specified item from the specified collection.
\begin{quote}\begin{description}
\item[{Parameters}] \leavevmode\begin{itemize}
\item {} 
\textbf{collection} (\emph{str}) -- a collection key

\item {} 
\textbf{item} (\emph{dict}) -- dict containing item data.

\end{itemize}

\item[{Return type}] \leavevmode
Boolean

\end{description}\end{quote}

See the {\hyperref[index:pyzotero.zotero.Zotero.delete_item]{\code{delete\_item()}}} example for multiple-item removal.

\end{fulllineitems}


\index{update\_collection() (pyzotero.zotero.Zotero method)}

\begin{fulllineitems}
\phantomsection\label{index:pyzotero.zotero.Zotero.update_collection}\pysiglinewithargsret{\code{Zotero.}\bfcode{update\_collection}}{\emph{collection}}{}
Update an existing collection name.
\begin{quote}\begin{description}
\item[{Parameters}] \leavevmode
\textbf{collection} (\emph{dict}) -- a dict containing collection data, previously retrieved using one of the Collections calls (e.g. {\hyperref[index:pyzotero.zotero.Zotero.collections]{\code{collections()}}}).

\item[{Return type}] \leavevmode
Boolean

\end{description}\end{quote}

Example:

\begin{Verbatim}[commandchars=\\\{\}]
\PYG{c}{\PYGZsh{} get existing collections, which will return a list of dicts}
\PYG{n}{c} \PYG{o}{=} \PYG{n}{zot}\PYG{o}{.}\PYG{n}{collections}\PYG{p}{(}\PYG{p}{)}
\PYG{c}{\PYGZsh{} rename the last collection created in the library}
\PYG{n}{c}\PYG{p}{[}\PYG{l+m+mi}{0}\PYG{p}{]}\PYG{p}{[}\PYG{l+s}{'}\PYG{l+s}{name}\PYG{l+s}{'}\PYG{p}{]} \PYG{o}{=} \PYG{l+s}{'}\PYG{l+s}{Whither Digital Humanities?}\PYG{l+s}{'}
\PYG{c}{\PYGZsh{} update collection name on the server}
\PYG{n}{zot}\PYG{o}{.}\PYG{n}{update\PYGZus{}collection}\PYG{p}{(}\PYG{n}{c}\PYG{p}{[}\PYG{l+m+mi}{0}\PYG{p}{]}\PYG{p}{)}
\end{Verbatim}

\end{fulllineitems}


\index{delete\_collection() (pyzotero.zotero.Zotero method)}

\begin{fulllineitems}
\phantomsection\label{index:pyzotero.zotero.Zotero.delete_collection}\pysiglinewithargsret{\code{Zotero.}\bfcode{delete\_collection}}{\emph{collection}}{}
Delete a collection from the Zotero library.
\begin{quote}\begin{description}
\item[{Parameters}] \leavevmode
\textbf{collection} (\emph{dict}) -- a dict containing collection data, previously retrieved using one of the Collections calls (e.g. {\hyperref[index:pyzotero.zotero.Zotero.collections]{\code{collections()}}}).

\item[{Return type}] \leavevmode
Boolean

\end{description}\end{quote}

See the {\hyperref[index:pyzotero.zotero.Zotero.delete_item]{\code{delete\_item()}}} example for ways to delete multiple collections.

\end{fulllineitems}

\end{quote}


\chapter{Notes}
\label{index:notes}
All Read API methods return \textbf{lists} of \textbf{dicts} or, in the case of tag methods, \textbf{lists} of \textbf{strings}. Most Write API methods return either \code{True} if successful, or raise an error. See \code{zotero\_errors.py} for a full listing of these.

\begin{notice}{warning}{Warning:}
URL parameters will supersede API calls which should return e.g. a single item: \code{https://api.zotero.org/users/436/items/ABC?start=50\&limit=10} will return 10 items beginning at position 50, even though \code{ABC} does not exist. Be aware of this, and don't pass URL parameters which do not apply to a given API method. This is a limitation/foible of the Zotero API, and there's nothing I can do about it.
\end{notice}


\chapter{License}
\label{index:license}
Pyzotero is licensed under the \href{http://www.gnu.org/licenses/gpl.html}{GNU GPL Version 3}  license, in line with Zotero's own license. Details can be found in the file \code{license.txt}.


\renewcommand{\indexname}{Python Module Index}
\begin{theindex}
\def\bigletter#1{{\Large\sffamily#1}\nopagebreak\vspace{1mm}}
\bigletter{p}
\item {\texttt{pyzotero.zotero}}, \pageref{index:module-pyzotero.zotero}
\end{theindex}

\renewcommand{\indexname}{Index}
\printindex
\end{document}
